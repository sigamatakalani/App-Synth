\documentclass[english]{article}

\usepackage[utf8]{inputenc}
\usepackage{graphicx}
\usepackage[margin=2.5cm]{geometry}
\usepackage{hyperref}
\hypersetup{
    colorlinks=true,
    linkcolor=blue,
    filecolor=magenta,      
    urlcolor=cyan,
}
 
\urlstyle{same}

\begin{document}
\begin{titlepage}
	\begin{center}
		\begin{figure}[t]
			\centering
			\includegraphics[width=350px]{logo.PNG}
		\end{figure}
		\begin{center}
			\textsc{\LARGE COS 301}
		\end{center}
		\begin{center}		
			\textsc{\LARGE V3D Graph Visualizer: User Manual}		
		\end{center}
		
		\begin{flushright} \large
			App-Synth \newline \emph{} \newline
		\end{flushright}


	\end{center}
\end{titlepage}


\newpage
\pagenumbering{arabic}
\thispagestyle{empty}
\tableofcontents
\clearpage

\setcounter{page}{1}

\section{Introduction}
\subsection{System Overview}
The V3D Digraph Visualizer is a graph visualization tool that allows users to perceive the information contained within digraphs which will be defined using the set of triples notations specified by Barla-Szabo in a 3D environment. The system will allow users to visualize and interact with the graph models in a 3D space. The system will consist of a mobile and a desktop application. The mobile application will allow the user to visualise the graph in a 3D space with the use of a mobile virtual reality headset, while the desktop application displays the visualisation for observers.

\subsection{System Configuration}
Unity game engine is used to model the visual components and interface of the system, with C Sharp being the primary programming language used. Unity 2017.1 for development can run on Windows 7, 8 and 10 and also from MAC OS X 10.9 moving forward to later developments. For the development of the graph ideally Visual Studio can be used for debugging purposes.

\subsubsection{Dependency Requirements (Development)} 

\begin{itemize}
	\item \textbf{GPU}: Graphics card with DX9 (shader model 3.0) or DX11 with feature level 9.3 capabilities.
	\item \textbf{Visual Studio}: 2012 or higher. \href{https://www.visualstudio.com/downloads/}{Download Here}
	\item \textbf{Android Studio}: Android SDK and Java Development Kit (JDK) \href{https://developer.android.com/studio/index.html}{Download Here}
	\item \textbf{Internet Connection}: Desktop needs to have an active Wi-Fi connection to facilitate communication with the mobile application.
	
\end{itemize}

\subsubsection{Dependency Requirements (Compiled App)}

\begin{itemize}
   \item A mobile phone running Android 4.1 (Jelly Bean) or newer.
   \item Google Cardboard VR headset or equivalent
   \item 25mb of memory. 
\end{itemize} 


\section{Installation and Getting Started}
Download the installation file onto the mobile device. Once the file has been downloaded, begin the installation. The android mobile device will request the user to grant access to the mobiles file system. This is necessary in order for the application to access the graph specification files needed to load a graph. These files will be stored on the device internal storage. Once this has been established, the application will be installed onto the device. 

To run the Desktop version, the user would have needed to download the (.exe) file of the application. Once downloaded, the user can run the installation and then proceed to open the desktop application once it has been installed. 

\section{Usage}
\begin{flushleft}
Once you have installed all the dependencies and have launched the Application on both the Desktop application and the mobile application, you will be presented with the following interfaces:

Once you have selected the graph which you want to visualize, you will be presented a visualisation of the graph in a display similar to the screenshot below.
\end{flushleft}
    
\begin{figure}
	\centering
	\includegraphics[width=350px]{"Images/Graph Visualization".png}
\end{figure}

\begin{flushleft}
As seen on the screen shot above, there is a cursor on the center of the screen in the form of a white dot. This cursor us used to interact with various buttons and controls in the VR environment. The cursors movement is controlled by the users head movements. In order to interact with a control, simply hover the cursor over the appropriate control. On the screen there are four arrows positioned on the top, left, right and bottom of the screen. These arrows are used to rotate the graph. There is also a back button located towards the top-left corner of the screen which is used to return to the main menu.
\end{flushleft}

\section{Troubleshooting}

\end{document}
